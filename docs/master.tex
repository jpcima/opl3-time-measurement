%%%%%%%%%%%%%%%%%%%%%%%%%%%%%%%%%%%%%%%%%
% Journal Article
% LaTeX Template
% Version 1.4 (15/5/16)
%
% This template has been downloaded from:
% http://www.LaTeXTemplates.com
%
% Original author:
% Frits Wenneker (http://www.howtotex.com) with extensive modifications by
% Vel (vel@LaTeXTemplates.com)
%
% License:
% CC BY-NC-SA 3.0 (http://creativecommons.org/licenses/by-nc-sa/3.0/)
%
%%%%%%%%%%%%%%%%%%%%%%%%%%%%%%%%%%%%%%%%%

%----------------------------------------------------------------------------------------
%	PACKAGES AND OTHER DOCUMENT CONFIGURATIONS
%----------------------------------------------------------------------------------------

\documentclass[twoside,twocolumn]{article}

\usepackage{blindtext} % Package to generate dummy text throughout this template

\usepackage[sc]{mathpazo} % Use the Palatino font
\usepackage[T1]{fontenc} % Use 8-bit encoding that has 256 glyphs
\usepackage[utf8]{inputenc}
\linespread{1.05} % Line spacing - Palatino needs more space between lines
\usepackage{microtype} % Slightly tweak font spacing for aesthetics

\usepackage[english]{babel} % Language hyphenation and typographical rules

\usepackage[hmarginratio=1:1,top=32mm,columnsep=20pt]{geometry} % Document margins
\usepackage[hang, small,labelfont=bf,up,textfont=it,up]{caption} % Custom captions under/above floats in tables or figures
\usepackage{booktabs} % Horizontal rules in tables

\usepackage{lettrine} % The lettrine is the first enlarged letter at the beginning of the text

\usepackage{enumitem} % Customized lists
\setlist[itemize]{noitemsep} % Make itemize lists more compact

\usepackage{abstract} % Allows abstract customization
\renewcommand{\abstractnamefont}{\normalfont\bfseries} % Set the "Abstract" text to bold
\renewcommand{\abstracttextfont}{\normalfont\small\itshape} % Set the abstract itself to small italic text

\usepackage{titlesec} % Allows customization of titles
\renewcommand\thesection{\Roman{section}} % Roman numerals for the sections
\renewcommand\thesubsection{\roman{subsection}} % roman numerals for subsections
\titleformat{\section}[block]{\large\scshape\centering}{\thesection.}{1em}{} % Change the look of the section titles
\titleformat{\subsection}[block]{\large}{\thesubsection.}{1em}{} % Change the look of the section titles

% \usepackage{fancyhdr} % Headers and footers
% \pagestyle{fancy} % All pages have headers and footers
% \fancyhead{} % Blank out the default header
% \fancyfoot{} % Blank out the default footer
% \fancyhead[C]{Running title $\bullet$ May 2016 $\bullet$ Vol. XXI, No. 1} % Custom header text
% \fancyfoot[RO,LE]{\thepage} % Custom footer text

\usepackage{titling} % Customizing the title section

\usepackage{hyperref} % For hyperlinks in the PDF

\usepackage{listings}
\usepackage{amsmath}

%----------------------------------------------------------------------------------------
%	TITLE SECTION
%----------------------------------------------------------------------------------------

\setlength{\droptitle}{-4\baselineskip} % Move the title up

\pretitle{\begin{center}\Huge\bfseries} % Article title formatting
\posttitle{\end{center}} % Article title closing formatting
\title{OPL3 envelope analysis} % Article title
\author{%
\textsc{Jean Pierre Cimalando}\thanks{} \\[1ex] % Your name
%\normalsize University of California \\ % Your institution
\normalsize \href{mailto:jp-dev@inbox.ru}{jp-dev@inbox.ru} % Your email address
%\and % Uncomment if 2 authors are required, duplicate these 4 lines if more
%\textsc{Jane Smith}\thanks{Corresponding author} \\[1ex] % Second author's name
%\normalsize University of Utah \\ % Second author's institution
%\normalsize \href{mailto:jane@smith.com}{jane@smith.com} % Second author's email address
}
\date{\today} % Leave empty to omit a date
\renewcommand{\maketitlehookd}{%
\begin{abstract}
\noindent This document describes the results obtained in attempts to determine approximations
for OPL3 envelope functions and their inverse, using a gray box experimental approach.
\end{abstract}
}

%----------------------------------------------------------------------------------------

\begin{document}

% Print the title
\maketitle

%----------------------------------------------------------------------------------------
%	ARTICLE CONTENTS
%----------------------------------------------------------------------------------------

\section{Introduction}

\lettrine[nindent=0em,lines=3]{T} he goal of this envelope analysis effort is to provide
good approximations of the key-on and key-off durations, in reasonable computation time.

Existing software realize this measurement operation ahead of time (ADLMIDI). As it is a
time consuming operation, it is not applicable to instruments which are modified in real time.

Experimental measurements are realized on Nuked OPL 1.8, a high fidelity emulator of the
Yamaha YMF262.

%------------------------------------------------

\section{Methods}

\subsection{Waveforms}

For the sake of level computations which will follow, we need a measurement of RMS levels
of the internal waves.

The internal waves, numbered 0—7, are computed in source code from \emph{EnvelopeCalcSin}
collection of functions. Running these functions in phase range 0—1023, and with envelope
0, produces a period of output normalized in the range $\pm 4096$.

\begin{table}
\begin{tabular}{llr}
\toprule
Waveform & RMS level \\
\midrule
Sine      & 0.705614      \\
Half-Sine      & 0.384713      \\
Abs-Sine      & 0.307053      \\
Pulse-Sine      & 0.307053      \\
Sine — even periods only      & 0.497982      \\
Abs-Sine — even periods only      & 0.384261      \\
Square      & 0.997680      \\
Derived-Square      & 0.214195      \\
\bottomrule
\end{tabular}
\caption{Waveform RMS levels}\label{tab:wave-rms}
\end{table}

The computed RMS levels are presented in~\ref{tab:wave-rms}.

\subsection{Envelope generalities}\label{sec:env-generalities}

For the envelope generator, we examine in detail the function \emph{EnvelopeCalc}.

A previous study~\cite{opl3analyzed} has given us a formula to determine an effective
envelope rate, based on several instrument parameters, and the frequency of the active note.

\begin{itemize}
  \item{ar, dr, rr:} rate of every non-stationary phase of the generator (0—15)
  \item{sl:} sustain level (0—15)
  \item{tl:} output attenuation (0—63)
  \item{ksr:} envelope scaling (0—1)
  \item{nts:} impact of note frequency on envelope scaling (0—1)
  \item{eg:} whether the generator has a sustain phase (0—1)
  \item{fnum, block:} representation of note frequency (0—1023, 0—7)
\end{itemize}

\begin{figure}
\begin{lstlisting}
unsigned eff_rate = 4 * ins_rate;
if (!ksr)
  eff_rate += block >> 1;
else
  eff_rate += (block << 1) |
    ((fnum >> (9 - nts)) & 1);
\end{lstlisting}
\caption{Effective envelope rate}\label{code:eff-rate}
\end{figure}

For the approximations of the envelope I make, for the notion of \emph{rate} I will
refer to the effective rate, as defined by the piece of C code~\ref{code:eff-rate}. The variable
\emph{ins\_rate} is to substitute with either \emph{ar}, \emph{dr} or \emph{rr},
depending on which phase of the envelope is under study.

\begin{figure}
\begin{lstlisting}
int tlv = tl << 2;
int kslv =
  ((kslrom[fnum >> 6] << 2) -
    ((8 - block) << 5)) >>
  kslshift[ksl];
\end{lstlisting}
\caption{Envelope level modifiers}\label{code:level-modifiers}
\end{figure}

Finally the envelope has two level modifiers which subtract from the output level,
depending on \emph{tl} and \emph{ksl}. These are determined by the C code~\ref{code:level-modifiers}.

The envelope generator produces its output in 0—511, but reversed. I will consider the
envelope output which is reversed back to normality and scaled to 0—1 range.

\subsection{Attack phase}\label{sec:attack-phase}

To examine a particular phase of envelope generation, I ran it under all possible
effective rates, resulting from all combinations of parameter values and note frequencies.

I wrote a program which took samples of envelope output, and was able to categorize the
samples into their envelope phases, by examining the value of the emulator for \emph{eg\_gen}.

I have converted the sample intervals into real time, by scaling down by the native sample
rate of 49716~Hz.

This gives a 2-parameter sampled function $(Time, Rate) \rightarrow Envelope$.

For the attack phase in particular, this function is an exponential shape.
I can find a decent approximation by fitting this function to:
\begin{equation}
E(t,r) = 1-\exp(-a b^{r+c} t)
\end{equation}
where $t$, $r$ are time and rate respectively, and $a$, $b$, $c$ are curve fitting constants
shown in~\ref{tab:env-attack-fit}.

\begin{table}
\begin{center}
\begin{tabular}{lr}
\toprule
a & 1.149780 \\
b & 1.189578 \\
c & -1.771204 \\
\midrule
R-square & 0.9956 \\
RMSE & 0.01699 \\
\bottomrule
\end{tabular}
\end{center}
\caption{Attack phase approximation}\label{tab:env-attack-fit}
\end{table}

\subsection{Release phase}\label{sec:release-phase}

By following exactly the same method as described in~\ref{sec:attack-phase}, I obtain a
sampled function for the release phase.

This one has linear shape and is approximated by the following function:
\begin{equation}
E(t,r) = 1-a b^{r+c} t
\end{equation}
where $t$, $r$ are time and rate respectively, and $a$, $b$, $c$ are curve fitting constants
shown in~\ref{tab:env-release-fit}.

\begin{table}
\begin{center}
\begin{tabular}{lr}
\toprule
a & 0.010613 \\
b & 1.185552 \\
c & 1.013799 \\
\midrule
R-square & 0.9895 \\
RMSE & 0.02902 \\
\bottomrule
\end{tabular}
\end{center}
\caption{Release phase approximation}\label{tab:env-release-fit}
\end{table}

\subsection{Estimating instrument durations}

Given a FM instrument for OPL3, I am interested in determining two particular values.

$t_{on}$ is defined as the duration for output amplitude to raise above or equal the target
amplitude value $V_t$ after key-on.

$t_{off}$ is defined as the duration for output amplitude to raise below or equal the target
amplitude value $V_t$ after key-off.

We have $E(t,r)$ as the envelope function defined over time and rate. In addition, I define
$E'(t,r)$ as the adjusted envelope function, subtracting level modifiers from envelope $E(t,r)$.
Level modifiers have been previously explained in~\ref{sec:env-generalities}, \ref{code:level-modifiers}.

The total amplitude level at some time point can be explained in words, as the combined
amplitude of all carriers under the AM modulation of their respective envelopes. In FM
synthesis a modulator has little impact on overall signal level, which is why I consider
only carriers to simplify.

If $N$ is the number of carriers, $R_i$ is the effective rate of the $i^{th}$ carrier, and $W_i$
is the RMS amplitude of the $i^{th}$ carrier's waveform, overall RMS level $V(t)$ is defined as:

\begin{equation}
  V(t) = \sqrt{\sum_{i=1}^{N} {(E'_i(t,R_i) W_i)}^2}
  \label{eq:rms-overall}
\end{equation}

In order to determine the duration we want, we have to substitute $V(t)$ for our desired
target level, and then solve for $t$.

\subsection{Key-on duration}

In order to solve~\ref{eq:rms-overall} we can use the analytical approach, but the
exponential formula for attack makes this difficult.

For a simple way to do it, we can repeatedly evaluate~\ref{eq:rms-overall}, and narrow
down $t$ with the binary search method.

With a range 0—10 for $t$ and 16 iterations, there is approximately a maximum
absolute error of $1.5 \times 10^{-4}$.

\subsection{Key-off duration}

We can solve this analytically, by plugging $E'(t,r)$ in~\ref{eq:rms-overall}.

To make the writing more concise, let's define $R'_i$ the slope of the $i^{th}$ carrier as
$a \times b^{r_i+c}$, and $\hat{V_i}$ the starting point of the $i^{th}$ carrier as $Vs_i - Vm_i$, where
$Vs_i$ is the sustain level and $Vm_i$ is the level modifier~\ref{sec:env-generalities}, \ref{code:level-modifiers}.

This gives:

\begin{align*}
  V(t) = \sqrt{\sum_{i=1}^{N} {((\hat{V_i} - R'_i t) W_i)}^2} \\
  V^2(t) = \sum_{i=1}^{N} {({R'_i}^2 W_i^2) t^2 - (2 {R'_i} \hat{V_i} W_i^2) t + \hat{V_i}^2 W_i^2}
\end{align*}

To determine $t$ we can have our direct answer by substituting $V(t)$ with the target level, and then
solving roots of the quadratic polynomial:

\begin{equation*}
(\sum_{i=1}^{N} {{R'_i}^2 W_i^2}) t^2 + (\sum_{i=1}^{N} {- 2 {R'_i} \hat{V_i} W_i^2}) t + (\sum_{i=1}^{N} {\hat{V_i}^2 W_i^2}) - V^2(t)
\end{equation*}

%----------------------------------------------------------------------------------------
%	REFERENCE LIST
%----------------------------------------------------------------------------------------

\begin{thebibliography}{99} % Bibliography - this is intentionally simple in this template

\bibitem[Steffen Ohrendorf, 2014]{opl3analyzed}
Steffen Ohrendorf. (2014).
\newblock OPL3 analyzed.
\newblock {\em \href{https://github.com/gtaylormb/opl3_fpga}{OPL3-FPGA}}

\end{thebibliography}

%----------------------------------------------------------------------------------------

\end{document}
